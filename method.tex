\chapter{Methodology}\label{ch:method}

This chapter describes the core of your work: the methodology you used to solve the problem
you described in the Introduction. It is quite likely to have a different name (e.g., the
name of the tool you wrote), and/or to be split into multiple chapters. It will likely be quite
densely connected with things you have written in the Background chapter (\Cref{ch:background}),
and the results that you will show in the Results chapter (\Cref{ch:results}). Put abundant
references to help the reader navigate the document.

Try to keep a level of detail such that a skilled reader will be able to reproduce your result,
but do not include unnecessary details that the reader can find out for themselves (e.g., the
list of commands to perform a standard task). That is indeed valuable information that can help
reproducibility, but if you do not think it is needed to understand your work, its place is
likely to be in an appendix or in the documentation of the code you wrote, if you decide to
release it.

This chapter should include work that has been done by you. If that is not the case, consider
moving it to the Background chapter (\Cref{ch:background}), or make it very clear that you are
not the author. The same thing applies to images: if they aren't yours, you should cite the
source.

The results of your work should generally not be included in this chapter. In some cases,
preliminary results can be included to explain why you took a certain direction in your work
(e.g., you took a choice rather than another because you measured that it was more efficient),
but the overall evaluation of your work should be in the Results chapter (\Cref{ch:results}).

You are likely to include algorithms in pseudocode. Check
\myhref{https://www.overleaf.com/learn/latex/Algorithms}{relevant \latex packages} for that.
