\chapter{Introduction}
\label{ch:introduction}

This is a template for the Computer Science Master's Theses at \myhref{https://dibris.unige.it}{\ac{DIBRIS}, University of Genoa.}
It uses the \xspace{\LaTeX} class \texttt{masterthesis.cls} by
Davide Ancona. We will use this template to discuss what is generally expected in the
structure of a Master's Thesis. While reading academic documents, you will find that
they generally follow a similar structure.

This structure is generally good for research documents, where your main goal is to do
something new. In other cases (e.g., the goal of your thesis is to review a topic and provide
an understanding of the field), a different structure may be more desirable. In any case,
discuss the structure of your thesis with your advisor.

I have put hints and links through the document for tools you can use; some may become stale over time, and something better may appear. If you find some that are not covered here, consider sending me an email or a pull request to integrate them!

\section{Style and Language}

We are writing a scientific document, and we should consider that skepticism
is at the core of the scientific method. We should be skeptical of our work, and
we can see the thesis as an effort to persuade a skeptical reader that we have done
a piece of work that is both useful and solid. Whenever you state something, you should give proof:
either by referencing one or more authoritative documents or proving it yourself.
Avoid marketing-speak (``our work is poised to revolutionize...'').\footnote{Unless you can
prove what you say. But these statements are mostly impossible to prove\ldots\ By the
way, notice that footnotes should be put after punctuation as in this case.}

Do not try to reach a given number of pages. Instead, try to make it
\emph{complete}, \emph{clear} and \emph{concise}. Try to give all the information that will
convince the reader that your work is solid and that you have done a good job; use simple
language and consistent terminology.
If you can say something more simply, do it. Avoid
\myhref{https://en.wikipedia.org/wiki/Weasel_word}{weasel words} and, when you can, the passive voice.
``We consider this to be not a security risk'' is much clearer than ``This is not considered a
security risk'' because it makes it explicit who makes the claim, and takes the responsibility for
it.

Consider that your reader will be a fellow computer scientist, but not necessarily an expert
on your topic. Try to give them the information they need to understand your work, but
do not lose your reader's time and attention with unnecessary details.

Documents like this generally follow a structure like this one. It is not a strict
requirement, but if you follow it readers will likely find it easier to navigate.
Consider that in scientific documents some readers will not read the whole document,
but will look for specific pieces of information. A clear structure will help them.

Chapters generally start with a short introduction to their topic. This is followed by the main content of the chapter, and then by a conclusion
that summarizes the main points. Note that both chapter and section titles are
\myhref{https://www.grammarly.com/blog/capitalization-in-the-titles/}{capitalized.}

Introduce the expanded form of acronyms the first time you use them, excepting only those so common that you can reasonably take for granted for any reader (e.g., CPU or PDF).

For this document, you should use formal writing. \myhref{https://www.grammarly.com/blog/contractions/}{Do not use contractions} such as \emph{don't} or \emph{it's}.

Be extremely wary of \acp{LLM} like Gemini, ChatGPT, or
Copilot.\footnote{Note that we used the serial comma (\url{https://en.wikipedia.org/wiki/Serial_comma}) here. It is a good idea to use it as it can remove ambiguity from your text.} You take
responsibility for the content of your thesis and declare you have written it yourself.
If you want a tool that helps you to write in better English, consider using something like
\myhref{https://grammarly.com}{Grammarly}; it is also \myhref{https://www.overleaf.com/learn/how-to/Use_Grammarly_with_Overleaf}{supported by the Overleaf online \latex editor.} Note that Overleaf now has a generative AI option: do not use it without declaring it, as it would violate its \myhref{https://www.grammarly.com/terms}{terms of service} which, among other things, forbids ``to mislead any third party that any output from Grammarly’s generative AI was solely human generated''.

\section{Motivation}

This is the first important thing you should write in your Introduction.
You can see it as the effort to convince your reader that the problem you are addressing
\emph{matters}. Mention that the problem is unsolved, but you will discuss that at
%length in the \fullref{ch:related}. Consider that
you're facing a skeptical reader asking ``is this useful?''.

Sometimes it will be obvious why the problem matters, but in other cases, this will be
one of the most delicate parts of your work. Of course, your advisor will help you with this.

\section{Content of the Thesis}

The other main goal of the Introduction is to give the reader an overview of what they will find in the document.
There is no concept of ``spoiler'' in scientific documents: summarize what the reader will find in every chapter. Avoid obvious statements that help nobody such as, for example, \emph{``In the Related Work chapter we discuss relevant pieces of work in the state of the art''}; provide synthetic but useful information instead (e.g., \emph{``In \cref{ch:related} we discuss other density-based clustering algorithms and highlight the differences to our approach'')}.
This will help them to navigate it (and to find what they're interested in, if they are reading the whole document).

I am not a fan of those \emph{``This document is structured as follows''} paragraphs.
My advice is to expose the content of the thesis and give references, organically, to the parts of it that treat them while you are at it.

The Introduction is arguably the most important chapter of an academic text because, after reading it, one should understand the key concepts of your work. The rest of the document ``only'' explains the details; you can expect somebody to stop at the Introduction alone.\footnote{Likewise, while you are going through somebody else's work, it makes sense to stop at the Introduction (and maybe the Conclusion) if you're interested only in a high-level overview.}

\section{\latex Tricks}

If you're using \latex to write your document, you can use some tricks to make your life
easier. This document uses the \verb|\Cref|%
\footnote{I also created a \texttt{\textbackslash fullref} to generate text like ``\fullref{ch:introduction}''. Check the source if you are interested.}
command to conveniently reference sections, figures, and tables, and \verb|\textcite| and
\verb|\authorcite| to cite paper authors. You can also use \verb|\xspace| to define your macros
and avoid spacing problems. Check the source code of this document to see how they are used.

\texttt{rubber} is a good tool to compile \latex documents. You can run the command
\texttt{rubber -d main} to compile this document to PDF, while running \texttt{pdflatex}
and \texttt{biber} (or \texttt{bibtex}) the right number of times.

The \myhref{https://www.overleaf.com}{Overleaf} online editor is a good tool to synchronize with
your advisor; it exports a git repository so you can edit your work online and offline.

To handle acronyms, the \myhref{https://ctan.org/pkg/acronym}{\texttt{acronym}} package is a good resource to automatically introduce an acronym's expanded form the first time you use it. It is also possible to create an acronyms table; consider using it if you use many acronyms in your paper and you think it may help the reader. Once again, you can have a look at how it's used in the source of this document.

You are likely going to cite several websites in your bibliography. Here is \myhref{https://bibtex.eu/faq/how-can-i-use-bibtex-to-cite-a-website/}{documentation} on how to do that in both BibTeX and BibLaTeX. You may want to use a reference management software like \myhref{https://www.zotero.org/}{Zotero} or \myhref{https://www.mendeley.com/}{Mendeley} to handle your bibliography database.

Your supervisors and reviewers will likely go over your document in several revisions; I'm sure they would really be thankful to you if you used a way to clearly show what changed since their previous pass. \myhref{https://ctan.org/pkg/latexdiff}{Latexdiff} is a tool to do just that; you can also \myhref{https://www.overleaf.com/learn/latex/Articles/How_to_use_latexdiff_on_Overleaf}{use it on Overleaf}. Another possibility is tracking the changes on the PDF output, using tools like \myhref{https://www.draftable.com/}{Draftable}.

Advisors may have very different preferences in how to share feedback. A possibility is the \myhref{https://tug.ctan.org/macros/latex/contrib/todonotes/todonotes.pdf}{\texttt{todonotes}} package. This document's \texttt{preamble.tex} source file shows an example of using it to include comments; note that they can be removed from the PDF output by passing the \texttt{final} option to the document class (i.e., \verb|\documentclass[final]{masterthesis}| in the main \texttt{tex} file.

\md{This is an example of a note using \texttt{todonotes}.}

%TODO
\textbf{IMPORTANT: Make sure the curriculum on the cover pages and the department address matches your own.} Currently, you'll have to edit the \texttt{masterthesis.cls} file accordingly; I plan to eventually provide a class option to facilitate that without the need to edit the \texttt{.cls} file. If you feel like you want to help me with that, please let me know! You'll get gratitude and an acknowledgment here.

