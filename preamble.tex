% we use biblatex, which is a more modern alternative to the traditional bibtex
% comment this to use bibtex instead
\usepackage[
    backend=biber,
    style=alphabetic,
    sorting=nyt,
    maxbibnames=99
]{biblatex}

% graphicx is needed to include images
\usepackage{graphicx}

% xspace is used to add a space after a macro, if needed
\usepackage{xspace}

% acronym (https://ctan.org/pkg/acronym) is a package to expand acronyms after their first usage
\usepackage{acronym}

% the hyperref package provides links; with this configuration they are colored
% according to the scheme described in https://tex.stackexchange.com/a/525297
\usepackage{xcolor}
\usepackage[colorlinks]{hyperref}
\def\tmp#1#2#3{%
  \definecolor{Hy#1color}{#2}{#3}%
  \hypersetup{#1color=Hy#1color}}
\tmp{link}{HTML}{800006}
\tmp{cite}{HTML}{2E7E2A}
\tmp{file}{HTML}{131877}
\tmp{url} {HTML}{8A0087}
\tmp{menu}{HTML}{727500}
\tmp{run} {HTML}{137776}
\def\tmp#1#2{%
  \colorlet{Hy#1bordercolor}{Hy#1color#2}%
  \hypersetup{#1bordercolor=Hy#1bordercolor}}
\tmp{link}{!60!white}
\tmp{cite}{!60!white}
\tmp{file}{!60!white}
\tmp{url} {!60!white}
\tmp{menu}{!60!white}
\tmp{run} {!60!white}

% the cleveref package is used to automatically add the type of reference
% (e.g., "section", "figure", "table") to the reference using the \cref command
% The "capitalize" option makes both \cref and \Cref uppercase; if we remove it
% only \Cref will be uppercase.
\usepackage[capitalize]{cleveref}

% the booktabs package is used to create better tables; check an example of usage in results.tex
\usepackage{booktabs}

%breaks long URLs better
\usepackage{xurl}

% macros
\newcommand*{\latex}{\LaTeX \xspace}

\usepackage{hyperref}
\newcommand{\myhref}[2]{#2\footnote{\url{#1}}}

% tracking changes using the todonotes package
\setlength {\marginparwidth }{2cm}
\usepackage{todonotes}
\newcommand{\note}[3]{\todo[inline,color=#2]{#1: #3}}
\newcommand{\md}[1]{\note{MD}{yellow}{#1}}

% result is something like "Related Work (Chapter 2)"
\newcommand{\fullref}[1]{\nameref{#1} (\Cref{#1})}

% used for the inparaenum environment
\usepackage{paralist}

