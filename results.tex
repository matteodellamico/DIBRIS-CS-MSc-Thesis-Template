\chapter{Results}
\label{sec:results}

Here you will present the results of your work. Your skeptical reader is now asking themself
``is this working?''. You will show here to what extent it does. Be honest about the limitations:
if your system doesn't work in some cases, you should say so (and explain where and why, if you
can). Like the Method chapter, also this chapter may end up being split into multiple chapters.

Keep in mind that this shouldn't be a mere dump of experimental results; you're rather
\emph{teaching} your reader how and to what extent you managed to solve the problem you
described in \Cref{sec:introduction}. If you have additional results that may be useful
but are not necessary to understand the points you're making (e.g., you evaluated your system
on multiple datasets and the results all tell the same story), the place for them is in an
appendix.

This chapter should have a lot of links to the methodology chapter(s), because you're
evaluating the choices you made there. If you developed a system made of multiple parts,
make sure that you test them separately and together, so that the reader can understand how
important each part is.

\begin{table}
    \centering
    \begin{tabular}{ccc}
        \toprule
        \textbf{Column 1} & \textbf{Column 2} & \textbf{Column 3} \\
        \midrule
        1 & 2 & 3 \\
        4 & 5 & 6 \\
        7 & 8 & 9 \\
        \bottomrule
    \end{tabular}
    \caption{A table using the \latex{} \texttt{booktabs} package. Note there are no vertical
    rules. In case you want to do more fancy stuff (e.g., merging cells), check also the
    \texttt{multirow} and \texttt{multicol} packages.}
    \label{tab:table}
\end{table}

This chapter is likely to be quite full of figures and tables. Try to make them as informative
as possible (e.g., use multiple lines in the same plot if possible). Showing graphs
effectively is a complex art; try to spend some time on it and ask for guidance from your
advisor. Whenever you have a figure or a table, make sure that you refer to it in the text
(after all, if it's not referred to in the text it means it has no part in the story you're
telling, so it has no place in this Chapter). Always use the text (and the caption, if you can
do it synthetically) to explain what you want the reader to understand from the figure.

For Figures, generally use vectorial formats (e.g., PDF, SVG) where it makes sense if you can.
They will result in higher-quality images that are in most cases also smaller, leading to
shorter compiling times and a smaller resulting PDF.

Tables look better without vertical rules: an example is in \Cref{tab:table}.

\section{Discussion}

At the end of this chapter, take a step back and summarize all the results so that the reader
can understand the big picture. Sometimes, this becomes large and important enough to be worth
a chapter on its own.
