\chapter{Related Work}
\label{sec:related}

This chapter is dedicated to the work that has been done by others in the field of your thesis.
Don't write a mere list of papers with a summary of each. Instead, the goal is to give a picture
of how your work is situated in the context of the work of others. In this case, you should
think your skeptical reader is asking ``hasn't this been solved already?''.

There will be two main kinds of work that you will need to discuss: work that is similar
to yours, and work that is complementary to yours.

Similar work will try to solve the same problem you are addressing, or a similar one.
Here, you should focus on the differences: for example, you may write something along the
lines of ``\textcite{DBLP:journals/x/Turing50} provides a naïve definition of
`intelligent machine', which has this and that limitation. Our work is similar to
\citeauthor{DBLP:journals/x/Turing50}, but we address these problems by doing this and
that''.\footnote{Don't be arrogant! I called Turing's work naïve as a joke.
For sure your work will have limitations that other work
may not have, and you should be honest about them.}

Complementary work does not try to solve the same problem you are addressing, but it is
important to get a full picture of your contribution. For example, if you are
\emph{not} attempting to solve a particular problem (it is ``out of scope'' for your work),
but the solution to that problem is necessary for your work to be useful, you should explain
that here.

While doing your review of related work, \href{https://scholar.google.com}{Google Scholar}
is probably the easiest search engine for academic papers. You can be mindful of some
bibliometric tricks to understand which papers are likely to be most influential in the field.
You can see on Scholar the number of citations of a paper;
\href{https://portal.core.edu.au/conf-ranks/}{CORE} and
\href{https://www.scimagojr.com}{Scimago} provide rankings of conferences and journals
respectively. Non-peer-reviewed articles like those published only on
\href{https://arxiv.org}{arXiv} may, of course, be important, but you should be more careful
and skeptical about them.

Another tool that you should \emph{use with caution} is \href{https://www.semanticscholar.org/}{Semantic Scholar}. It provides you with \ac{LLM}-based tools to synthesize and ask questions about papers. Always double-check, because \acp{LLM} hallucinate and, as usual, you take responsibility for whatever you write in your document: treat \acp{LLM} as a useful tool you shouldn't trust.

Wikipedia is a great source of information, but you shouldn't consider it as reliable.
Wikipedia articles should always provide the sources of the information they contain; you
should check it yourself, judge its reliability, and cite the original source.

In general, be careful about citing websites: make sure that they're authoritative, and when it's not obvious justify why you consider them to be worth citing. Unlike this document, which I don't expect readers to print, your thesis must be readable in printed form: hence, make URLs readable in the text.

If a paper is very important with respect to your own piece of work, have a look at papers
that cited it, to see if there are more recent works that are relevant to your thesis.

Here are a few tricks about how to cite other pieces of work:
\begin{itemize}
  \item If a paper is available in a peer-reviewed venue (journal or conference), cite that
  version rather than the non-reviewed one. Sometimes papers are published in two similar
  versions: one in a conference and an extended one in a journal. In that case, cite the
  journal.
  \item Use citations such as ``The Hamiltonian cycle problem is
  NP-complete~\cite{DBLP:conf/coco/Karp72}'' or ``\textcite{DBLP:conf/coco/Karp72} showed
  that the Hamiltonian cycle problem is NP-complete''. Don't write
  ``\cite{DBLP:conf/coco/Karp72} showed that the Hamiltonian cycle problem is NP-complete''.
  \item \href{https://dblp.org}{DBLP} is a good source of quality Bibtex records for papers
  you may want to cite.
  \item You are likely to cite a lot of work that is published on the Web.
  \href{https://bibtex.eu/faq/how-can-i-use-bibtex-to-cite-a-website/}{Here} is some
  documentation about how to cite a website.
\end{itemize}
