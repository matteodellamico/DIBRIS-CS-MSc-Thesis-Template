\chapter{Background}
\label{ch:background}

This chapter provides the necessary background information to understand the content of your thesis.
The goal is to make your thesis \emph{self-contained}: the reader, who you can assume will be
a fellow Computer Science student, should be able to understand your work without having to refer
to external sources. Of course, you should reference the sources you used to write this chapter, and references to documents that the reader can use to know more about those topics.

Do not include information that is not necessary to understand your work. If something is complementary to your work, but not useful to understand it, you can include it in the \nameref{ch:related} (\Cref{ch:related}).

Guide the reader while they are going through this chapter: do not just dump information, but present the context in which they will be used in the thesis, and what those concepts will be useful for.\footnote{A good exercise is thinking that your reader will constantly ask ``so what?'' after everything you tell them. If it is unclear why you provided them with information, you should explain (or, of course, remove the content if there is no explanation).}
Put forward references to the places these concepts are used, and put backward references from that chapter to here so that the reader can link the pieces.\footnote{Always keep in mind that the reader will not necessarily read all your documents. Such forward- and back-references help refresh the memory of those who read your documents linearly \emph{and} help those who do not.}
Pay close attention, here and everywhere else in your thesis, to make sure that every non-trivial concept you use is defined clearly before being used.

Sometimes, there will be concepts that are tangential to your work: in those cases, avoid having an unclear or incomplete explanation: either explain it clearly in sufficient depth, or skip the concept altogether, leaving a bibliographic reference for those who want to understand the issue. A short but unclear explanation that will confuse readers is worse than either choice.

The Background chapter contains information about work that \emph{you didn't do}. The work you
did should be described in the \nameref{ch:method} (\Cref{ch:method}) instead.

The amount of information you will need to present in this chapter can vary a lot between theses.
In some cases, you will have no or very little need for a Background chapter; in that case, it
may make sense to move this information to a section in the Related Work chapter.
